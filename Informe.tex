\documentclass[11pt,oneside,a4paper]{article}
\usepackage{xltxtra} % Para usar características avanzadas de XeLaTeX, como elegir la fuente.
\usepackage{minted} 
\usepackage{multicol}
\usepackage{polyglossia} % Un sustituto de babel.
\setdefaultlanguage{spanish}
\usepackage{indentfirst} % Necesario para que sangre la primera línea de los primeros párrafos.
 
\defaultfontfeatures{Mapping=tex-text,Ligatures=Common} % Para que funcionen cosas como --- y las ligaduras.

\usepackage{hyperref}
%\usepackage{showframe}
\usepackage[top=2in, bottom=1.5in, left=1.5in, right=1in]{geometry}
%\usepackage{}
%\setmainfont{CMU Serif} % Ésta hay que descargarla. Si no, puedes usar la Linux Libertine.
%\setsansfont{CMU Sans Serif}
%\setmonofont{CMU Typewriter Text}
%\setmainfont{Linux Libertine 0}
%\setsansfont{Linux Biolinum O}
%\setmonofont{Inconsolata}


%\usepackage[pdftex]{graphicx}

\newcommand{\HRule}{\rule{\linewidth}{0.5mm}}

%\renewcommand\thesection{\arabic{section}}

\title{FDisk}
\author{Lauro Figueroa}
\date{1 de Enero de 1970}
\begin{document}
 
\begin{titlepage}
\begin{center}

% Upper part of the page. The '~' is needed because \\
% only works if a paragraph has started.
%\includegraphics[width=0.15\textwidth]{./Logo.jpg}~\\[1cm]

\textsc{\LARGE Facultad de Ciencias Exactas, Ingeniería y Agrimensura}\\[1.5cm]
	
\textsc{\Large Trabajo de cátedra}\\[0.5cm]

% Title
\HRule \\[0.4cm]
{ \huge \bfseries Sistema Experto \\[0.4cm] }

\HRule \\[1.5cm]

% Author and supervisor
\noindent
\begin{minipage}{0.4\textwidth}
\begin{flushleft} \large
\emph{Alumnos:}\\
Emilio \textsc{Lopez}\\
Lauro \textsc{Figueroa}
\end{flushleft}
\end{minipage}%
\begin{minipage}{0.4\textwidth}
\begin{flushright} \large
\emph{Profesores:} \\
Ana \textsc{Casali}\\
Federico \textsc{Guimpel}\\
\end{flushright}
\end{minipage}

\vfill

% Bottom of the page
{\large \today}

\end{center}
\end{titlepage}

%\maketitle
\newpage 

\section*{Introducción}

El sistema experto propuesto clasifica juegos, los mismos como 
dominio nos permiten generar un procedimiento que busque y verifique
una hipótesis generando preguntas solo cuando son necesarias e informe con          % o.O
"Te recomendamos jugar: ..." sobre que elemento del dominio el
usuario puede estar interesado.


\section{Conceptualizar el dominio}

Los objetos del dominio son:

\begin{multicols}{2}
\begin{itemize}
\item Counter Strike: Global Offensive,
\item Grand Theft Auto: V,
\item DOTA 2,
\item Garry's Mod,
\item Cities Skyline,
\item The Sims 3,
\item Goat Simulator,
\item Team Fortress 2,
\item Half-Life 2,
\item DiRT 3,
\item Football Manager,
\item Portal 2,
\item Dying Light,
\item Kerbal Space Program y
\item Worms: Armaggedon.
\end{itemize}
\end{multicols}

Cada elemento de dominio tiene atributos que representan sus propiedades, las
mismas permiten clasificarlos. Por ejemplo Counter Strike: Global Offensive
es un juego muliplataforma, multijugador, fps, de acción y un shooter. En prolog 
lo representamos de la siguiente forma:\\

csgo(X) :- multiplataforma(X), multiplayer(X), fps(X), accion(X), shooter(X).

Donde el predicado shooter es representado: 

shooter(X) :- satisface('de tipo shooter',X).


\subsection{Relación entre los los atributos de un objeto y las reglas}

%Predicados. que son y como los combinamos?
%los predicados son reglas?, si lo son como se las carateriza o diferencia?

Todo sistema experto tiene una base de datos, un conocimiento previo sobre el
cual razonar. Las reglas de clasificación de los objetos son propiedades
sobre los mismos de las cuales podemos inferir conocimiento. Las reglas 
implementadas son:

\inputminted[firstline=17, lastline=31, fontsize=\footnotesize]{perl}{tp-prolog.pl}

Cada una de estas reglas de clasificacion esta compuesta por predicados.
Estas reglas por si solas no son más que conocimiento, pero necesitamos inferir 
sobre el mismo; es decir, necesitamos un mecanismo sobre el cual interpretar las preferencias 
de un usuario y agregar ese nuevo conocimento a la base de datos.

% falta algo Titulo?
\subsection{Expandir la base de datos}

El predicado \emph{pregunta} dado en clase nos da una forma de ganar nuevo conocimiento
realizando preguntas al usuario sobre sus preferencias.
Estas permiten generar un sistema interactivo y generar nuevo %hacen interactiva la experiencia al usuario
conocimiento a medida que ciertas preguntas son respondidas y sus respectivas respuestas
agregadas a la base de datos.\\

% meh - borrar 
%Pero, ¿Qué son las 
%preguntas?, las preguntas son la unica manera que tiene el sistema diseñado 
%para ganar conocimiento sobre lo que el usuario desea. ¿Que pasa cuando las responde?
%cada vez que el usuario interactivamente las responde el 
%conjunto de conocimiento se agranda, expandiendo el nivel de exactitud  % meh
%de con que la respuesta sera entregada.
% expandiendo el poder de inferencia


%¿Cómo inferir nuevo conocimiento? ¿Cómo lo guardo?
Todo conocimiento ganado es almacenado.
Cada pregunta puede ser respondida con sí o no (o una leve variación 
sintáctica de alguna de las mismas). El sistema hace una pregunta 
sólo cuando desconoce su respuesta; es decir, sabe que elemento del dominio 
posee un atributo o no, lo que desconoce es en que conjuntos de atributos 
el usuario quiere que se satisfagan. En el siguiente ejemplo el 
sistema desconoce si el usuario esta interesado en juegos para Windows, por lo
que pregunta:\\

¿Te interesa un juego que corra en Windows?\\

De responder si agrega si('que corrar en Windows') a la base de datos usando 
assert de manera analoga si hubiera respondido no. Lo que hay que remarcar en ambos casos 
es que la base de datos se expandio, ganando un criterio o preferencia del usuario.

\subsection{Combinando reglas} % borrar

El problema de clasificación por etapas está presente en este sistema, el predicado 
multiplataforma esta compuesto por tres predicados más simples
combinados por un \emph{o} lógico que obliga al inferidor que proceda en etapas, 
categorías que luego se especializan en Windows, GNU/Linux o MacOS. 

%\subsection{Interpretacion y resultados}

\newpage
\section{Ejemplo completo de ejecución}

El siguiente es un ejemplo de una consulta exitosa, a la cual es sistema encontro
solucion:

?- juego.
¿Te interesa un juego que corra en Windows? si.

¿Te interesa un juego que sea multijugador? si.

¿Te interesa un juego de tipo FPS? si.

¿Te interesa un juego de acción? si.

¿Te interesa un juego de estrategia? no.

¿Te interesa un juego de tipo shooter? si.

Te recomendamos jugar: Counter Strike: Global Offensive

Aunque hay otras a las cuales el sistema no puede dar solución:

?- juego.
¿Te interesa un juego que corra en Windows? no.

¿Te interesa un juego que corra en Mac OS X? no.

¿Te interesa un juego que corra en GNU/Linux? no.

Te recomendamos jugar: no sabemos :(


\end{document}

%windows(X) :- satisface(para\_windows,X).\\
%macos(X) :- satisface(para\_macos,X).\\
%linux(X) :- satisface(para\_linux,X).\\
%multiplataforma(X) :- windows(X); macos(X); linux(X).

%multiplayer(X) :- satisface(es\_multiplayer,X).\\
%singleplayer(X) :- satisface(es\_singleplayer,X).

%fps(X) :- satisface(tipo\_fps,X).\\\begin{tabular}{lll}
%accion(X) :- satisface(tipo\_accion,X).\\
%shooter(X) :- satisface(tipo\_shooter,X).\\
%estrategia(X) :- satisface(tipo\_estrategia,X).\\
%simulacion(X) :- satisface(tipo\_simulacion,X).\\x
