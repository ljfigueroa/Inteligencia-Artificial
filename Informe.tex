\documentclass[11pt,oneside,a4paper]{article}
\usepackage{xltxtra} % Para usar características avanzadas de XeLaTeX, como elegir la fuente.
\usepackage{minted} 
\usepackage{polyglossia} % Un sustituto de babel.
\setdefaultlanguage{spanish}
\usepackage{indentfirst} % Necesario para que sangre la primera línea de los primeros párrafos.
 
\defaultfontfeatures{Mapping=tex-text,Ligatures=Common} % Para que funcionen cosas como --- y las ligaduras.

\usepackage{hyperref}
%\usepackage{showframe}
\usepackage[top=2in, bottom=1.5in, left=1.5in, right=1in]{geometry}
%\usepackage{}
%\setmainfont{CMU Serif} % Ésta hay que descargarla. Si no, puedes usar la Linux Libertine.
%\setsansfont{CMU Sans Serif}
%\setmonofont{CMU Typewriter Text}
%\setmainfont{Linux Libertine 0}
%\setsansfont{Linux Biolinum O}
%\setmonofont{Inconsolata}


%\usepackage[pdftex]{graphicx}

\newcommand{\HRule}{\rule{\linewidth}{0.5mm}}

%\renewcommand\thesection{\arabic{section}}

\title{FDisk}
\author{Lauro Figueroa}
\date{1 de Enero de 1970}
\begin{document}
 
\begin{titlepage}
\begin{center}

% Upper part of the page. The '~' is needed because \\
% only works if a paragraph has started.
\includegraphics[width=0.15\textwidth]{./Logo.jpg}~\\[1cm]

\textsc{\LARGE Facultad de Ciencias Exactas, Ingeniería y Agrimensura}\\[1.5cm]

\textsc{\Large Trabajo final}\\[0.5cm]

% Title
\HRule \\[0.4cm]
{ \huge \bfseries Sistema Experto \\[0.4cm] }

\HRule \\[1.5cm]

% Author and supervisor
\noindent
\begin{minipage}{0.4\textwidth}
\begin{flushleft} \large
\emph{Alumno:}\\
Emilio \textsc{Lopez}\\
Lauro \textsc{Figueroa}
\end{flushleft}
\end{minipage}%
\begin{minipage}{0.4\textwidth}
\begin{flushright} \large
\emph{Profesores:} \\
Dr.Esteban \textsc{Ruiz}\\
Dr.Fededico \textsc{Bergero}\\
Dr.Diego \textsc{Feroldi}
\end{flushright}
\end{minipage}

\vfill

% Bottom of the page
{\large \today}

\end{center}
\end{titlepage}

%\maketitle
\newpage 

\section*{Introducción}

El sistema experto propuesto clasifica juegos, los mismos como 
dominio nos permite generar un procedimiento que busque y verifique
una hipótesis generando preguntas necesarias e informe con
"Te recomendamos jugar: ..." sobre que elemento del dominio uno 
puede estar interezado.


\section{Conseptualizar el dominio}

Los objetos del dominio son:
csgo, 
gta5, 
dota2, 
garrys mod, 
cities skyline, 
sims3, 
goat simulator, 
tf2, 
hl2, 
dirt3, 
football manager, 
portal2, 
dying light, 
kerbal space program, 
l4d2 y 
worms armaggedon.

Las reglas de clasificación son:

\begin{itemize}
  \item csgo es un multiplataforma, multijugador, fps, shooter de accion. 
  \item gta5 es para windows, multijugador de estrategia.
  \item 
  \item 
  \item 
%  \item 
%  \item 
%  \item 
%  \item 
%  \item 
%  \item 
%  \item 
%  \item 
%  \item 
%  \item 
%  \item 
\end{itemize}

\subsection{Relación entre los los atibutos de un objeto y las reglas}

%Predicados. que son y como los combinamos?
%los predicados son reglas?, si lo son como se las carateriasa o diferencia?

Las reglas de clasificacion de los objetos son


\inputminted[firstline=1, lastline=13, fontsize=\footnotesize]{perl}{tp-prolog.pl}


\subsection{Combinando reglas}



\subsection{Interpretacion y resultados}

\section{Implementar el sistema experto}

\section*{Bibliografia}


\href{http://susam.in/articles/boot-sector-code/}{Susam In}\\
\href{http://www.pelletiernet.com/helppc/int_13-2.html}{INT 13}\\
\href{http://en.wikipedia.org/wiki/Master_boot_record}{MBR wiki}

\end{document}

%windows(X) :- satisface(para\_windows,X).\\
%macos(X) :- satisface(para\_macos,X).\\
%linux(X) :- satisface(para\_linux,X).\\
%multiplataforma(X) :- windows(X); macos(X); linux(X).

%multiplayer(X) :- satisface(es\_multiplayer,X).\\
%singleplayer(X) :- satisface(es\_singleplayer,X).

%fps(X) :- satisface(tipo\_fps,X).\\
%accion(X) :- satisface(tipo\_accion,X).\\
%shooter(X) :- satisface(tipo\_shooter,X).\\
%estrategia(X) :- satisface(tipo\_estrategia,X).\\
%simulacion(X) :- satisface(tipo\_simulacion,X).\\x
