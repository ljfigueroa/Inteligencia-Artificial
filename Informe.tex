\documentclass[11pt,oneside,a4paper]{article}
\usepackage{xltxtra} % Para usar características avanzadas de XeLaTeX, como elegir la fuente.
\usepackage{minted} 
\usepackage{multicol}
\usepackage{polyglossia} % Un sustituto de babel.
\setdefaultlanguage{spanish}
\usepackage{indentfirst} % Necesario para que sangre la primera línea de los primeros párrafos.
 
\defaultfontfeatures{Mapping=tex-text,Ligatures=Common} % Para que funcionen cosas como --- y las ligaduras.

\usepackage{hyperref}
%\usepackage{showframe}
\usepackage[top=2in, bottom=1.5in, left=1.5in, right=1in]{geometry}
%\usepackage{}
%\setmainfont{CMU Serif} % Ésta hay que descargarla. Si no, puedes usar la Linux Libertine.
%\setsansfont{CMU Sans Serif}
%\setmonofont{CMU Typewriter Text}
%\setmainfont{Linux Libertine 0}
%\setsansfont{Linux Biolinum O}
%\setmonofont{Inconsolata}


%\usepackage[pdftex]{graphicx}

\newcommand{\HRule}{\rule{\linewidth}{0.5mm}}

%\renewcommand\thesection{\arabic{section}}

\title{FDisk}
\author{Lauro Figueroa}
\date{1 de Enero de 1970}
\begin{document}
 
\begin{titlepage}
\begin{center}

% Upper part of the page. The '~' is needed because \\
% only works if a paragraph has started.
%\includegraphics[width=0.15\textwidth]{./Logo.jpg}~\\[1cm]

\textsc{\LARGE Facultad de Ciencias Exactas, Ingeniería y Agrimensura}\\[1.5cm]

\textsc{\Large Trabajo de cátedra}\\[0.5cm]

% Title
\HRule \\[0.4cm]
{ \huge \bfseries Sistema Experto \\[0.4cm] }

\HRule \\[1.5cm]

% Author and supervisor
\noindent
\begin{minipage}{0.4\textwidth}
\begin{flushleft} \large
\emph{Alumnos:}\\
Emilio \textsc{Lopez}\\
Lauro \textsc{Figueroa}
\end{flushleft}
\end{minipage}%
\begin{minipage}{0.4\textwidth}
\begin{flushright} \large
\emph{Profesores:} \\
Ana \textsc{Casali}\\
Federico \textsc{Guimpel}\\
\end{flushright}
\end{minipage}

\vfill

% Bottom of the page
{\large \today}

\end{center}
\end{titlepage}

%\maketitle
\newpage 

\section*{Introducción}

El sistema experto propuesto clasifica juegos, los mismos como 
dominio nos permiten generar un procedimiento que busque y verifique
una hipótesis generando preguntas "necesarias" e informe con          % o.O
"Te recomendamos jugar: ..." sobre que elemento del dominio el
usuario puede estar interesado.


\section{Conceptualizar el dominio}

Los objetos del dominio son:

\begin{multicols}{2}
\begin{itemize}
\item Counter Strike: Global Offensive,
\item Grand Theft Auto: V,
\item DOTA 2,
\item Garry's Mod,
\item Cities Skyline,
\item The Sims 3,
\item Goat Simulator,
\item Team Fortress 2,
\item Half-Life 2,
\item DiRT 3,
\item Football Manager,
\item Portal 2,
\item Dying Light,
\item Kerbal Space Program,
\item Left 4 Dead 2 y
\item Worms: Armaggedon.
\end{itemize}
\end{multicols}

Las reglas de clasificación son:

\begin{itemize}
  \item csgo es un multiplataforma, multijugador, fps, shooter de accion. 
  \item gta5 es para windows, multijugador de estrategia.
  \item 
  \item 
  \item 
%  \item 
%  \item 
%  \item 
%  \item 
%  \item 
%  \item 
%  \item 
%  \item 
%  \item 
%  \item 
%  \item 
\end{itemize}

\subsection{Relación entre los los atributos de un objeto y las reglas}

%Predicados. que son y como los combinamos?
%los predicados son reglas?, si lo son como se las carateriza o diferencia?

Todo sistema experto tiene una base de datos, un conocimiento previo sobre el
cual razonar. Las reglas de clasificación de los objetos son propiedades
sobre los mismos de las cuales podemos inferir conocimiento.

\inputminted[firstline=1, lastline=13, fontsize=\footnotesize]{perl}{tp-prolog.pl}

Estas reglas por si solas no son más que conocimiento, pero necesitamos inferir 
sobre el mismo; es decir, necesitamos un mecanismo sobre el cual interpretar las preferencias 
de un usuario y agregar ese nuevo conocimento a la base de datos.

% falta algo Titulo?

El predicado \emph{pregunta} dado en clase nos da una forma de ganar nuevo conocimiento
realizando preguntas al usuario sobre sus preferencias.
Permiten generar un sistema interactivo con el usuario y generar nuevo
conocimiento a medida que ciertas preguntas son respondidas y agregarlo a
la base de datos.

% meh - borrar 
%Pero, ¿Qué son las 
%preguntas?, las preguntas son la unica manera que tiene el sistema diseñado 
%para ganar conocimiento sobre lo que el usuario desea. ¿Que pasa cuando las responde?
%cada vez que el usuario interactivamente las responde el 
%conjunto de conocimiento se agranda, expandiendo el nivel de exactitud  % meh
%de con que la respuesta sera entregada.
% expandiendo el poder de inferencia


¿Cómo inferir nuevo conocimiento? ¿Cómo se guarda el mismo?

Cada pregunta puede ser respondida con sí o no (o una leve variación 
sintáctica de alguna de las mismas). El sistema hace una pregunta 
sólo cuando desconoce su respuesta; es decir, sabe que elemento del dominio 
posee un atributo o no, lo que desconoce es en que conjuntos de atributos 
el usuario quiere que se satisfagan.

Implementamos usando satisface  % ?? chamuyo 
provista en clase,% su funcionamiento es simple si un el atributo buscado no 

\subsection{Combinando reglas} % borrar

%multiplataforma

El predicado multiplataforma, esta compuesto por tres predicados más simples
combinados por un \emph {o} lógico que 

\subsection{Interpretacion y resultados}

\section{Implementar el sistema experto}


\section*{Bibliografia}


\href{http://susam.in/articles/boot-sector-code/}{Susam In}\\
\href{http://www.pelletiernet.com/helppc/int_13-2.html}{INT 13}\\
\href{http://en.wikipedia.org/wiki/Master_boot_record}{MBR wiki}

\end{document}

%windows(X) :- satisface(para\_windows,X).\\
%macos(X) :- satisface(para\_macos,X).\\
%linux(X) :- satisface(para\_linux,X).\\
%multiplataforma(X) :- windows(X); macos(X); linux(X).

%multiplayer(X) :- satisface(es\_multiplayer,X).\\
%singleplayer(X) :- satisface(es\_singleplayer,X).

%fps(X) :- satisface(tipo\_fps,X).\\
%accion(X) :- satisface(tipo\_accion,X).\\
%shooter(X) :- satisface(tipo\_shooter,X).\\
%estrategia(X) :- satisface(tipo\_estrategia,X).\\
%simulacion(X) :- satisface(tipo\_simulacion,X).\\x
